\documentclass{article}

\usepackage{siunitx}
\usepackage{amsmath}
\usepackage{mathtools}
% \usepackage{algorithm}
% \usepackage{algpseudocode}
% \usepackage{minted}
\usepackage{booktabs}
\usepackage{url}
\frenchspacing

% \usemintedstyle{xcode}

\title{Project 2:\\Solving Schr\"odinger's equation for two electrons in a 3D harmonic oscillator well}
\author{Josh Bradt}
\date{March 4, 2016}

\begin{document}

\maketitle

\section{Introduction}
    A commonly studied system in physics is that of the electron confined in a potential well of some sort. This arises in, for example, the study of quantum dots, quantum computing, and other areas of solid-state physics.

    Assuming spherical symmetry, we can write the radial part of Schr\"odinger's equation as follows for the confined electron in three dimensions:
    \begin{equation}
        -\frac{\hbar^2}{2m} \left( \frac{1}{r^2} \frac{d}{dr} r^2 \frac{d}{dr} - \frac{l(l+1)}{r^2} \right) R(r) + V(r)R(r) = ER(r).  \label{eq:genschro}
    \end{equation}
    Here, $R(r)$ is the radial wave function of the electron, $l$ is the angular momentum, $V(r)$ is the confining potential, and $E$ is the energy. If we take the potential to be the three-dimensional harmonic oscillator potential,
    \begin{equation}
        V(r) = \frac{1}{2} m \omega^2 r^2,
    \end{equation}
    then the energies are known to be
    \begin{equation}
        E_{nl} = \hbar\omega \left( 2n + l + \frac{3}{2} \right)
    \end{equation}
    with quantum numbers $n=0,1,2,\dots$ and $l=0,1,2,\dots$.

    Next, make the substitution $R(r) = (1/r) u(r)$ in (\ref{eq:genschro}) to find
    \begin{equation*}
        -\frac{\hbar^2}{2m} \left( \frac{d^2}{dr^2} - \frac{l(l+1)}{r^2} \right) u(r) + V(r)u(r) = Eu(r).
    \end{equation*}
    This 



\end{document}
